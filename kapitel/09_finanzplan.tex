\chapter{Finanzplan}

\section{Kapitalbedarfsplan}

\subsection{Einmalige Kosten (Gründung \& Ausstattung)}
\begin{tabular}{lr}
    \toprule
    \textbf{Position} & \textbf{Betrag} \\
    \midrule
    Gewerbeanmeldung GbR & 50 \euro{} \\
    Notarkosten (Gesellschaftsvertrag) & 500 \euro{} \\
    Büroausstattung (3 Arbeitsplätze) & 15.000 \euro{} \\
    Hardware (3$\times$ Entwickler-PCs, Server) & 12.000 \euro{} \\
    Software-Lizenzen (IDE, Design-Tools) & 3.000 \euro{} \\
    Domain \& SSL-Zertifikate & 200 \euro{} \\
    \midrule
    \textbf{Summe einmalige Kosten} & \textbf{30.750 \euro{}} \\
    \bottomrule
\end{tabular}

\subsection{Laufende Kosten (24 Monate)}
\begin{tabular}{lr}
    \toprule
    \textbf{Position} & \textbf{Betrag (24 Mon.)} \\
    \midrule
    Büromiete inkl. Nebenkosten (24 $\times$ 2.000 \euro{}) & 48.000 \euro{} \\
    Gehälter (3 Gründer, 24 $\times$ 4.000 \euro{} $\times$ 3) & 288.000 \euro{} \\
    Hosting \& Cloud-Infrastruktur (24 $\times$ 200 \euro{}) & 4.800 \euro{} \\
    Marketing \& Werbung & 12.000 \euro{} \\
    Buchhaltung/Steuerberater & 3.600 \euro{} \\
    Versicherungen & 2.400 \euro{} \\
    Sonstiges (Reisen, Fortbildung) & 6.000 \euro{} \\
    \midrule
    \textbf{Summe laufende Kosten} & \textbf{364.800 \euro{}} \\
    \bottomrule
\end{tabular}

\subsection{Liquiditätsreserve}
25.000 \euro{}

\subsection{Gesamtkapitalbedarf}
\begin{tabular}{lr}
    \toprule
    \textbf{Kategorie} & \textbf{Betrag} \\
    \midrule
    Einmalige Kosten & 30.750 \euro{} \\
    Laufende Kosten (24 Monate) & 364.800 \euro{} \\
    Liquiditätsreserve & 25.000 \euro{} \\
    \midrule
    \textbf{Gesamtkapitalbedarf} & \textbf{ca. 375.000 \euro{}} \\
    \bottomrule
\end{tabular}

\section{Finanzierungsplan}

\subsection{Eigenkapitalanteil}
75.000 \euro{} (25.000 \euro{} pro Gründer aus Ersparnissen)

\subsection{Fremdkapitalbedarf}
\begin{tabular}{lr}
    \toprule
    \textbf{Quelle} & \textbf{Betrag} \\
    \midrule
    Bankdarlehen & 200.000 \euro{} \\
    Förderdarlehen (KfW) & 100.000 \euro{} \\
    \midrule
    \textbf{Summe Fremdkapital} & \textbf{300.000 \euro{}} \\
    \bottomrule
\end{tabular}

\subsection{Förderprogramme}
\begin{itemize}
    \item \textbf{EXIST-Gründerstipendium:} Für innovative technologiebasierte Gründungen
    \item \textbf{KfW-Gründerkredit StartGeld:} Bis 125.000 \euro{} ohne Eigenkapitalnachweis
    \item \textbf{ERP-Gründerkredit Universell:} Für größere Investitionen
    \item \textbf{Gründungszuschuss:} Falls aus Anstellung gegründet wird
    \item \textbf{Regionale Digitalbonus-Programme:} Je nach Bundesland
\end{itemize}

\subsection{Beteiligungskapitalgeber}
\begin{itemize}
    \item \textbf{Business Angels:} Im Bereich Nachhaltigkeit/FoodTech
    \item \textbf{High-Tech Gründerfonds (HTGF):} Für innovative Startups
    \item \textbf{Venture Capital:} Bei Skalierungsbedarf nach Jahr 2
    \item \textbf{Crowdinvesting:} Zur Einbindung der Community
\end{itemize}

\subsection{Leasing-Optionen}
\begin{tabular}{llr}
    \toprule
    \textbf{Objekt} & \textbf{Laufzeit} & \textbf{Rate/Monat} \\
    \midrule
    Büromöbel & 36 Monate & ca. 300 \euro{} \\
    Hardware (IT-Leasing) & 36 Monate & ca. 400 \euro{} \\
    \bottomrule
\end{tabular}

\vspace{0.3cm}
Option: Sale-and-Lease-Back bei Liquiditätsbedarf

\section{Sicherheiten}
\begin{itemize}
    \item Bürgschaften der drei Gesellschafter
    \item Verpfändung von Gesellschaftsanteilen
    \item Ggf. Bürgschaft durch Bürgschaftsbank des Landes
    \item Forderungsabtretung aus Kundenverträgen (ab Jahr 2)
\end{itemize}

\section{Liquiditätsplan}

\subsection{Monatliche Einzahlungen (3-Jahres-Prognose)}
\begin{tabular}{lll}
    \toprule
    \textbf{Zeitraum} & \textbf{Phase} & \textbf{Einnahmen/Monat} \\
    \midrule
    Jahr 1, Monat 1--6 & Entwicklung & 0 \euro{} \\
    Jahr 1, Monat 7--12 & Soft-Launch & ca. 2.000 \euro{} \\
    Jahr 2 & Wachstum & ca. 8.000 \euro{} \\
    Jahr 3 & Skalierung & ca. 25.000 \euro{} \\
    \bottomrule
\end{tabular}

\subsection{Monatliche Kosten}
\textbf{Fixkosten pro Monat:}
\begin{tabular}{lr}
    \toprule
    \textbf{Position} & \textbf{Betrag} \\
    \midrule
    Gehälter (3 Gründer) & 12.000 \euro{} \\
    Büromiete & 2.000 \euro{} \\
    Hosting/Cloud & 200--1.000 \euro{} \\
    Buchhaltung & 150 \euro{} \\
    Versicherungen & 100 \euro{} \\
    \midrule
    \textbf{Summe Fixkosten} & ca. 14.500 \euro{} \\
    \bottomrule
\end{tabular}

\vspace{0.3cm}
\textbf{Variable Kosten:}
\begin{itemize}
    \item Marketing: 500--2.000 \euro{}/Monat
    \item Sonstiges: 200 \euro{}/Monat
\end{itemize}

\textbf{Geschätzte Gesamtkosten: 15.000--17.000 \euro{}/Monat}

\subsection{Investitionskosten (erste 12 Monate)}
\begin{tabular}{lr}
    \toprule
    \textbf{Zeitraum} & \textbf{Investition} \\
    \midrule
    Monat 1 & 30.750 \euro{} (Gründung, Ausstattung) \\
    Monat 2--6 & je 1.000 \euro{} (Software, Tools) \\
    Monat 7--12 & je 2.000 \euro{} (Marketing, Server) \\
    \midrule
    \textbf{Gesamt Jahr 1} & \textbf{ca. 48.000 \euro{}} \\
    \bottomrule
\end{tabular}

\subsection{Kapitaldienst (Tilgung und Zinsen)}
\begin{tabular}{llr}
    \toprule
    \textbf{Darlehen} & \textbf{Konditionen} & \textbf{Rate/Monat} \\
    \midrule
    Bankdarlehen (200.000 \euro{}) & 5\% Zins, 10 Jahre & ca. 2.100 \euro{} \\
    KfW-Darlehen (100.000 \euro{}) & 2\% Zins, tilgungsfrei J. 1--2 & ab J. 3: 1.100 \euro{} \\
    \midrule
    \textbf{Gesamt} & & \textbf{2.100--3.200 \euro{}} \\
    \bottomrule
\end{tabular}

\subsection{Liquiditätsreserve}
\begin{tabular}{ll}
    \toprule
    \textbf{Jahr} & \textbf{Liquidität} \\
    \midrule
    Jahr 1 & Negativ (Aufbauphase) -- gedeckt durch Startkapital \\
    Jahr 2 & ca. -7.000 \euro{}/Monat $\rightarrow$ Break-Even Ende Jahr 2 \\
    Jahr 3 & ca. +5.000 \euro{}/Monat $\rightarrow$ Positiver Cashflow \\
    \bottomrule
\end{tabular}

\section{Ertragsvorschau / Rentabilitätsvorschau}

\subsection{Umsatzprognose (3 Jahre)}
\begin{tabular}{lr}
    \toprule
    \textbf{Jahr} & \textbf{Umsatz} \\
    \midrule
    Jahr 1 & ca. 12.000 \euro{} (Soft-Launch) \\
    Jahr 2 & ca. 96.000 \euro{} (Wachstum) \\
    Jahr 3 & ca. 300.000 \euro{} (Skalierung + B2B) \\
    \bottomrule
\end{tabular}

\subsection{Kostenprognose (3 Jahre)}
\begin{tabular}{lr}
    \toprule
    \textbf{Jahr} & \textbf{Kosten} \\
    \midrule
    Jahr 1 & ca. 230.000 \euro{} (Aufbau + Vollbetrieb) \\
    Jahr 2 & ca. 200.000 \euro{} (Vollbetrieb) \\
    Jahr 3 & ca. 220.000 \euro{} (+ Marketing) \\
    \bottomrule
\end{tabular}

\subsection{Gewinnprognose (3 Jahre)}
\begin{tabular}{lr}
    \toprule
    \textbf{Jahr} & \textbf{Gewinn/Verlust} \\
    \midrule
    Jahr 1 & ca. -218.000 \euro{} (Anlaufverlust) \\
    Jahr 2 & ca. -104.000 \euro{} (Verlustreduktion) \\
    Jahr 3 & ca. +80.000 \euro{} (erster Gewinn) \\
    \bottomrule
\end{tabular}

\subsection{Break-Even-Analyse}
\begin{itemize}
    \item \textbf{Break-Even:} Voraussichtlich Mitte Jahr 3
    \item \textbf{Kumulierter Verlust bis Break-Even:} ca. 300.000 \euro{}
    \item \textbf{Deckung:} Durch Eigenkapital (75.000 \euro{}) + Fremdkapital (300.000 \euro{})
\end{itemize}

\section{Finanzierungswissen}

\textbf{Genutzte Informationsquellen:}
\begin{itemize}
    \item IHK Gründungsberatung und Seminare
    \item Förderdatenbank des BMWi (foerderdatenbank.de)
    \item Gründerplattform.de
    \item KfW-Beratungsgespräch
    \item Startup-Workshops und Networking-Events
    \item Fachliteratur zu SaaS-Geschäftsmodellen
\end{itemize}
