\chapter{Zusammenfassung}

\textit{Stellen Sie Ihrem Businessplan -- nachdem Sie alle Inhalte zusammengetragen haben -- eine Zusammenfassung voran. Beschreiben Sie darin kurz und prägnant Ihre Geschäftsidee, die wesentlichen Erfolgs- und Risikofaktoren und Ihre Ziele für die ersten Jahre Ihrer Selbständigkeit.}

\section{Name des Unternehmens}
\textbf{FreshGuard} -- Intelligente Lebensmittelverwaltung

\section{Namen der Gründer}
\begin{itemize}
    \item Maike Abel
    \item Phillipp Atzler
    \item Elisa Dörnbrak
\end{itemize}

\section{Geschäftsidee}
Webbasierte Anwendung zur Bestands- und Haltbarkeitsverwaltung von Lebensmitteln für Haushalte und Wohngemeinschaften. Nutzer können Artikel mit Mindesthaltbarkeitsdatum (MHD) erfassen und werden rechtzeitig an ablaufende Produkte erinnert.

\section{Besonderheit der Geschäftsidee}
\begin{itemize}
    \item Mehrere Haushaltsmitglieder können gemeinsam einen Haushalt verwalten
    \item Automatische Warnungen bei bald ablaufenden Produkten (farbliche Kennzeichnung)
    \item Keine Installation nötig -- läuft im Browser auf allen Geräten
    \item Kostenloser Basis-Plan mit Premium-Optionen (Basic/Pro)
    \item Datenschutzkonform mit Serverstandort Deutschland
\end{itemize}

\section{Qualifikationen der Gründer}
Fachinformatiker Anwendungsentwicklung mit Kenntnissen in PHP, Webentwicklung, Datenbankdesign und Projektmanagement.

\section{Zielkunden}
Private Haushalte, Wohngemeinschaften und Familien, die ihre Lebensmittelvorräte besser verwalten und Verschwendung reduzieren möchten.

\section{Vertriebsweg}
Online-Marketing über Social Media, SEO-Optimierung und Empfehlungen durch zufriedene Nutzer.

\section{Kapitalbedarf}
\begin{tabular}{lr}
    \toprule
    \textbf{Position} & \textbf{Betrag} \\
    \midrule
    Büro + Miete, Internet, Strom (24 Monate) & 24.000 \euro{} \\
    Ausrüstung (Schreibtische, PCs, Monitore, Peripherie) & 4.300 \euro{} \\
    Arbeitsstunden (3 Pers. $\times$ 160h/Mon. $\times$ 24 Mon. $\times$ 30\euro{}) & 345.600 \euro{} \\
    Server-Hosting (24 Monate) & 900 \euro{} \\
    \midrule
    \textbf{Gesamt (24 Monate)} & \textbf{ca. 375.000 \euro{}} \\
    \bottomrule
\end{tabular}

\section{Erwartetes Umsatzvolumen}
\begin{tabular}{lr}
    \toprule
    \textbf{Jahr} & \textbf{Umsatz} \\
    \midrule
    Jahr 1 & ca. 5.000 \euro{} (ca. 100 Premium-Nutzer) \\
    Jahr 2 & ca. 15.000 \euro{} (ca. 300 Premium-Nutzer) \\
    Jahr 3 & ca. 30.000 \euro{} (ca. 600 Premium-Nutzer) \\
    \bottomrule
\end{tabular}

\section{Mitarbeiter in den ersten drei Jahren}
3 Gründer + ggf. 1--2 zusätzliche Entwickler/Support-Mitarbeiter

\section{Ziele}
\begin{itemize}
    \item Reduzierung der Lebensmittelverschwendung bei Nutzern um 15\%
    \item Rechtzeitiger Verbrauch von mindestens 85\% der erfassten Produkte
    \item Kontinuierliche Verbesserung basierend auf Nutzerfeedback
\end{itemize}

\section{Risiken}
\begin{itemize}
    \item Konkurrenz durch bestehende Apps
    \item Zu wenige zahlende Kunden für Premium-Pläne
    \item Technische Ausfälle oder Sicherheitsprobleme
\end{itemize}

\section{Geplanter Start}
\textbf{01. Februar 2026}
