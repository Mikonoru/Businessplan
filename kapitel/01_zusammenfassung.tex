\chapter{Zusammenfassung}

\textbf{FreshGuard -- Intelligente Lebensmittelverwaltung} ist eine webbasierte Anwendung zur Bestands- und Haltbarkeitsverwaltung von Lebensmitteln, gegründet von Maike Abel, Phillipp Atzler und Elisa Dörnbrak. Das Gründerteam besteht aus ausgebildeten Fachinformatikern der Anwendungsentwicklung mit fundierten Kenntnissen in PHP, Webentwicklung, Datenbankdesign und Projektmanagement.

\section{Geschäftsidee und Besonderheiten}

FreshGuard ermöglicht es Haushalten und Wohngemeinschaften, ihre Lebensmittelvorräte digital zu verwalten. Nutzer können Artikel mit Mindesthaltbarkeitsdatum erfassen und werden rechtzeitig an ablaufende Produkte erinnert. Die Besonderheit liegt in der Möglichkeit, dass mehrere Haushaltsmitglieder gemeinsam einen Haushalt verwalten können, wobei automatische Warnungen bei bald ablaufenden Produkten durch farbliche Kennzeichnung angezeigt werden. Als browserbasierte Anwendung ist keine Installation erforderlich -- FreshGuard läuft auf allen Geräten. Das Freemium-Modell bietet einen kostenlosen Basis-Plan mit Premium-Optionen (Basic/Pro), während die Daten datenschutzkonform auf Servern in Deutschland gespeichert werden.

\section{Zielkunden und Vertrieb}

Die Zielgruppe umfasst private Haushalte, Wohngemeinschaften und Familien, die ihre Lebensmittelvorräte besser verwalten und Verschwendung reduzieren möchten. Der Vertrieb erfolgt über Online-Marketing via Social Media, SEO-Optimierung und Empfehlungen zufriedener Nutzer.

\section{Kapitalbedarf und Umsatzerwartung}

Der Gesamtkapitalbedarf für 24 Monate beträgt circa 375.000 Euro und setzt sich zusammen aus Bürokosten inklusive Miete, Internet und Strom (24.000 Euro), Ausrüstung für Schreibtische, PCs, Monitore und Peripherie (4.300 Euro), Arbeitsstunden für drei Personen (345.600 Euro) sowie Server-Hosting (900 Euro).

\begin{tabular}{lr}
    \toprule
    \textbf{Jahr} & \textbf{Erwarteter Umsatz} \\
    \midrule
    Jahr 1 & ca. 5.000 \euro{} (ca. 100 Premium-Nutzer) \\
    Jahr 2 & ca. 15.000 \euro{} (ca. 300 Premium-Nutzer) \\
    Jahr 3 & ca. 30.000 \euro{} (ca. 600 Premium-Nutzer) \\
    \bottomrule
\end{tabular}

\section{Team und Ziele}

Das Team besteht aus den drei Gründern, wobei bei entsprechendem Wachstum wenige zusätzliche Entwickler oder Support-Mitarbeiter hinzukommen sollen. Die Kernziele sind die Reduzierung der Lebensmittelverschwendung bei Nutzern um 15 Prozent, der rechtzeitige Verbrauch von mindestens 85 Prozent der erfassten Produkte sowie die kontinuierliche Verbesserung basierend auf Nutzerfeedback.

\section{Risiken und Zeitplan}

Die größten Risiken bestehen in der Konkurrenz durch bestehende Apps, einer möglicherweise zu geringen Anzahl zahlender Kunden für Premium-Pläne sowie technischen Ausfällen oder Sicherheitsproblemen. Der geplante Marktstart ist der \textbf{01. Februar 2026}.
