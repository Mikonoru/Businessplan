\chapter{Marketing}

\section{Angebot und Kundennutzen}

FreshGuard bietet seinen Nutzern einen vollständigen Überblick über alle Lebensmittelvorräte im Haushalt mit rechtzeitiger Erinnerung an ablaufende Produkte durch farbliche Warnungen. Dies führt zu weniger Lebensmittelverschwendung und damit zu Geldersparnis. Die gemeinsame Verwaltung für alle Haushaltsmitglieder mit einem echten Multi-User-System inklusive Rollen (Eigentümer, Co-Eigentümer, Mitwirkende) unterscheidet FreshGuard von der Konkurrenz. Die DSGVO-Konformität mit deutschem Serverstandort und die Tatsache, dass keine Installation nötig ist, da alles im Browser funktioniert, runden das Angebot ab.

Der Service umfasst einen Online-Hilfe- und FAQ-Bereich, E-Mail-Support für alle Nutzer sowie Priority-Support für Pro-Kunden. Regelmäßige Updates und neue Features werden kontinuierlich bereitgestellt. Als Garantieleistungen werden eine 30-tägige Geld-zurück-Garantie bei Premium-Abos, Datensicherung und eine Verfügbarkeitsgarantie von 99 Prozent Uptime sowie die Möglichkeit zum kostenlosen Datenexport jederzeit angeboten.

\section{Preisstrategie}

FreshGuard verfolgt ein Freemium-Modell mit kostenlosem Basis-Plan und Premium-Upgrades. Der Free-Plan ermöglicht Einzelnutzern das Testen der Anwendung, während bezahlte Pläne erweiterte Funktionen und mehr Haushalte sowie Mitglieder bieten.

\begin{tabular}{llr}
    \toprule
    \textbf{Plan} & \textbf{Leistungen} & \textbf{Preis} \\
    \midrule
    Free & 1 Haushalt, keine Mitglieder & 0 \euro{} \\
    Basic & 2 Haushalte, 4 Mitglieder & 4,99 \euro{}/Monat \\
    Basic (Jahr) & 2 Haushalte, 4 Mitglieder & 44,99 \euro{}/Jahr \\
    Pro & 10 Haushalte, unbegrenzte Mitglieder & 9,99 \euro{}/Monat \\
    Pro (Jahr) & 10 Haushalte, unbegrenzte Mitglieder & 89,99 \euro{}/Jahr \\
    Pro Lifetime & Alle Pro-Features, einmalig & 199,99 \euro{} \\
    \bottomrule
\end{tabular}

\vspace{0.5cm}
Die laufenden Kosten für Hosting betragen circa 72 Euro pro Jahr und für die Domain circa 12 Euro pro Jahr, sodass der Break-Even bereits ab 10 zahlenden Nutzern erreicht wird. Die Preise orientieren sich an der Konkurrenz, wo 2 bis 5 Euro pro Monat üblich sind.

\section{Vertrieb}

Die Vertriebskosten sind minimal, da der Vertrieb vollständig als Self-Service erfolgt. Nutzer registrieren sich direkt über die Webanwendung und Premium-Upgrades erfolgen innerhalb der Anwendung. Es gibt keine Zwischenhändler oder Vertriebspartner. Die jährlichen Kosten belaufen sich auf circa 60 Euro für Serverkosten und 12 Euro für die Domain.

\begin{tabular}{lrrr}
    \toprule
    \textbf{Jahr} & \textbf{Registrierte Nutzer} & \textbf{Premium-Anteil} & \textbf{Premium-Nutzer} \\
    \midrule
    Jahr 1 & 100 & 10\% & 10 \\
    Jahr 2 & 500 & 15\% & 75 \\
    Jahr 3 & 1.500 & 20\% & 300 \\
    \bottomrule
\end{tabular}

\vspace{0.5cm}
Das Zielgebiet ist die DACH-Region (Deutschland, Österreich, Schweiz).

\section{Werbung}

Kunden erfahren von FreshGuard über Social Media (Instagram, Facebook, TikTok), SEO-Optimierung für Suchmaschinen, Mundpropaganda durch zufriedene Nutzer sowie Content-Marketing über Blog und Ratgeber. Ab dem Launch wird die Social-Media-Präsenz aufgebaut, in den Monaten eins bis drei folgt Content-Marketing mit Blog-Artikeln zu Lebensmittelverschwendung. In den Monaten drei bis sechs ist gezielte Social-Media-Werbung geplant und ab Monat sechs sollen Influencer-Kooperationen im Bereich Nachhaltigkeit hinzukommen.

\begin{tabular}{lr}
    \toprule
    \textbf{Jahr} & \textbf{Werbebudget} \\
    \midrule
    Jahr 1 & ca. 2.000 \euro{} \\
    Jahr 2 & ca. 5.000 \euro{} \\
    Jahr 3 & ca. 10.000 \euro{} \\
    \bottomrule
\end{tabular}

\vspace{0.5cm}
Der Marketing-Mix ist darauf ausgelegt, virales Wachstum durch das Freemium-Modell zu ermöglichen. Zufriedene Free-User werben neue Nutzer an und das Premium-Upgrade ist innerhalb der App prominent platziert. Das Nachhaltigkeits-Thema spricht gezielt die umweltbewusste Zielgruppe an.
