\chapter{Markt und Wettbewerb}

\section{Zielkunden}

\subsection{Wer sind Ihre Kunden?}
Private Haushalte und Wohngemeinschaften, die ihre Lebensmittelvorräte digital verwalten möchten.

\subsection{Wo sind Ihre Kunden?}
Primär deutschsprachiger Raum (DACH-Region), da Serverstandort Deutschland und deutschsprachige Benutzeroberfläche.

\subsection{Kundensegmente}
\begin{tabular}{ll}
    \toprule
    \textbf{Kriterium} & \textbf{Zielgruppe} \\
    \midrule
    Alter & 25--55 Jahre (technikaffin, umweltbewusst) \\
    Haushaltsgrößen & 2--6 Personen (Paare, Familien, WGs) \\
    Kundentyp & Privatkunden (B2C) \\
    Fokus & Nachhaltigkeit und bewusster Konsum \\
    \bottomrule
\end{tabular}

\subsection{Referenzkunden}
Testnutzer aus dem persönlichen Umfeld während der Entwicklungsphase.

\subsection{Abhängigkeit von Großkunden}
Nein, das Geschäftsmodell basiert auf vielen kleinen Einzelkunden (B2C). Keine Abhängigkeit von einzelnen Großkunden.

\subsection{Kundenbedürfnisse und Probleme}
\begin{itemize}
    \item Fehlender Überblick über Lebensmittelvorräte
    \item Vergessene Mindesthaltbarkeitsdaten
    \item Lebensmittelverschwendung und damit verbundene Kosten
    \item Mangelnde Koordination innerhalb des Haushalts
\end{itemize}

\subsection{Warum soll der Kunde bei Ihnen kaufen?}
\begin{itemize}
    \item Speziell für Haushalte mit mehreren Personen entwickelt
    \item Einfache Bedienung ohne Installation
    \item DSGVO-konform mit Serverstandort Deutschland
    \item Faire Preisgestaltung mit kostenlosem Einstieg
\end{itemize}

\subsection{Erwartete Umsätze}
\begin{tabular}{lr}
    \toprule
    \textbf{Jahr} & \textbf{Umsatz} \\
    \midrule
    Jahr 1 & ca. 12.000 \euro{} \\
    Jahr 2 & ca. 96.000 \euro{} \\
    Jahr 3 & ca. 300.000 \euro{} \\
    \bottomrule
\end{tabular}

\section{Wettbewerber}

\subsection{Konkurrierende Entwicklungen}
Ja, diverse Vorrats-Apps existieren am Markt:
\begin{itemize}
    \item Bring! (Einkaufslisten-fokussiert)
    \item NoWaste (ähnliches Konzept)
    \item Our Groceries
    \item Pantry Check
    \item Diverse kleinere MHD-Tracker-Apps
\end{itemize}

\subsection{Preise der Konkurrenz}
Meist kostenlose Basis-Version, Premium-Features zwischen 2--5 \euro{}/Monat.

\subsection{Stärken und Schwächen der Konkurrenten}
\begin{tabular}{p{6cm}p{7cm}}
    \toprule
    \textbf{Stärken} & \textbf{Schwächen} \\
    \midrule
    Etablierte Nutzerbasis & Oft keine echte Haushalts-Kollaboration \\
    App-Store-Präsenz & Kein deutsches Hosting (DSGVO-Bedenken) \\
    Native Apps verfügbar & Eingeschränkte Multi-User-Funktionen \\
    \bottomrule
\end{tabular}

\subsection{Worin unterscheiden Sie sich von Ihren Mitbewerbern?}
\begin{itemize}
    \item \textbf{Echte Multi-User-Haushalte:} Rollensystem mit Eigentümer, Co-Eigentümer und Mitwirkenden
    \item \textbf{Browserbasiert:} Keine Installation, plattformunabhängig
    \item \textbf{DSGVO-konform:} Serverstandort Deutschland
    \item \textbf{Transparente Preise:} Freemium-Modell mit fairen Premium-Optionen
\end{itemize}

\subsection{Eigene Schwächen gegenüber Konkurrenten}
\begin{tabular}{p{5cm}p{8cm}}
    \toprule
    \textbf{Schwäche} & \textbf{Gegenmaßnahme} \\
    \midrule
    Keine native App (nur Web) & Progressive Web App (PWA) für App-ähnliches Erlebnis \\
    Keine etablierte Nutzerbasis & Fokus auf Alleinstellungsmerkmal: echte Multi-User-Haushalte \\
    Keine App-Store-Präsenz & SEO und Social Media Marketing \\
    \bottomrule
\end{tabular}

\section{Standort}

\subsection{Standort des Unternehmens}
Online als Webanwendung, erreichbar über jeden Browser weltweit.

\subsection{Bedeutung des Standorts}
Als rein digitales Produkt ist der physische Standort des Unternehmens sekundär. Der Server-Standort (Deutschland) ist hingegen wichtig für Datenschutz und Vertrauen.

\subsection{Vorteile des Standorts}
\begin{itemize}
    \item Maximale Reichweite ohne Installations-Hürden
    \item Plattformunabhängig (Windows, Mac, Linux, iOS, Android)
    \item Skalierbar ohne geografische Einschränkungen
    \item Serverstandort Deutschland = Vertrauen bei DACH-Kunden
\end{itemize}

\subsection{Nachteile des Standorts}
\begin{tabular}{p{5cm}p{8cm}}
    \toprule
    \textbf{Nachteil} & \textbf{Ausgleichsmaßnahme} \\
    \midrule
    Internetverbindung erforderlich & PWA mit Service Worker für eingeschränkte Offline-Nutzung (geplant) \\
    Keine Offline-Funktionalität & Mobile-optimiertes responsives Design \\
    \bottomrule
\end{tabular}
