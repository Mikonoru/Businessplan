\chapter{Geschäftsidee: Produkt/Dienstleistung}

\section{Zweck des Vorhabens}
Haushalten und Wohngemeinschaften eine einfache Lösung zu bieten, um Lebensmittelvorräte zu verwalten und Verschwendung durch vergessene Mindesthaltbarkeitsdaten zu reduzieren.

\section{Beschreibung des Produkts/der Dienstleistung}
\textbf{FreshGuard} -- Webportal zur Bestands- und Haltbarkeitsverwaltung mit:
\begin{itemize}
    \item Artikelerfassung (Name, MHD, Menge, Lagerort)
    \item Haushalts-Verwaltung mit Einladungssystem
    \item Dashboard mit Statistiken und Warnungen
    \item Lizenz-System (Free/Basic/Pro)
\end{itemize}

\section{Besonderheit der Geschäftsidee (USP)}
\begin{itemize}
    \item \textbf{Multi-User-Haushaltsverwaltung:} Rollensystem mit Eigentümer, Co-Eigentümer und Mitwirkenden
    \item \textbf{Automatische Warnungen:} Farbliche Kennzeichnung (Grün/Gelb/Rot) bei ablaufenden Produkten
    \item \textbf{Browserbasiert:} Keine Installation nötig, responsive auf allen Geräten
    \item \textbf{Datenschutz:} DSGVO-konform mit Serverstandort Deutschland
\end{itemize}

\section{Kundennutzen}
\begin{itemize}
    \item Weniger Lebensmittelverschwendung durch rechtzeitige Erinnerungen
    \item Besserer Überblick über vorhandene Vorräte
    \item Gemeinsame Verwaltung für alle Haushaltsmitglieder
    \item Kostenersparnis durch optimierten Verbrauch
    \item Zeitersparnis bei der Einkaufsplanung
\end{itemize}

\section{Unternehmensziele}

\subsection{Kurzfristige Ziele (Jahr 1)}
\begin{itemize}
    \item MVP fertigstellen und in Betrieb nehmen
    \item Erste Nutzer gewinnen
    \item Feedback sammeln und einarbeiten
    \item 100+ aktive Nutzer erreichen
\end{itemize}

\subsection{Langfristige Ziele (Jahr 2--5)}
\begin{itemize}
    \item Nutzerbasis auf 1.000+ aktive Haushalte ausbauen
    \item Premium-Funktionen erweitern:
    \begin{itemize}
        \item Barcode-Scan zur schnellen Artikelerfassung
        \item E-Mail-Erinnerungen an ablaufende Produkte
        \item Rezeptvorschläge basierend auf vorhandenen Zutaten
    \end{itemize}
    \item Mobile App entwickeln (iOS/Android)
\end{itemize}

\section{Entwicklungsstand}
MVP (Minimal Viable Product) mit allen Kernfunktionen fertiggestellt:
\begin{itemize}
    \item \textbf{Phase 1:} Basisstruktur -- \textit{abgeschlossen}
    \item \textbf{Phase 2:} Haushalts-Verwaltung -- \textit{abgeschlossen}
    \item \textbf{Phase 3:} Lizenz-System -- \textit{abgeschlossen}
    \item \textbf{Phase 4:} Admin-Panel -- \textit{abgeschlossen}
\end{itemize}

\section{Voraussetzungen bis zum Start}
\begin{itemize}
    \item Produktiv-Hosting einrichten
    \item Finale Tests durchführen
    \item Benutzerdokumentation vervollständigen
    \item Marketingmaterialien erstellen
\end{itemize}

\section{Preisstrategie}
\begin{tabular}{llr}
    \toprule
    \textbf{Plan} & \textbf{Features} & \textbf{Preis} \\
    \midrule
    Free & 1 Haushalt, 50 Artikel, 1 Nutzer & 0 \euro{}/Monat \\
    Basic & 3 Haushalte, unbegrenzte Artikel, 5 Nutzer & 4,99 \euro{}/Monat \\
    Pro & Unbegrenzt, alle Features, Priority-Support & 9,99 \euro{}/Monat \\
    Pro Lifetime & Alle Pro-Features, einmalige Zahlung & 199,99 \euro{} \\
    \bottomrule
\end{tabular}

\section{Start der Vermarktung}
\textbf{01. Februar 2026} nach Abschluss der Testphase

\section{Gesetzliche Formalitäten}
\begin{itemize}
    \item Gewerbeanmeldung
    \item Impressum erstellen
    \item Datenschutzerklärung (DSGVO-konform)
    \item Allgemeine Geschäftsbedingungen (AGB) formulieren
\end{itemize}

\section{Patente und Schutzrechte}
\begin{itemize}
    \item Markenname \glqq FreshGuard\grqq{} beim DPMA schützen lassen (geplant)
    \item Domain freshguard.de gesichert
    \item Quellcode als Geschäftsgeheimnis geschützt
\end{itemize}
