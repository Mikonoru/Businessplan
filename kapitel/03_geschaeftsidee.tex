\chapter{Geschäftsidee: Produkt/Dienstleistung}

\section{Produktbeschreibung und Kundennutzen}

FreshGuard ist ein Webportal zur Bestands- und Haltbarkeitsverwaltung von Lebensmitteln, das Haushalten und Wohngemeinschaften eine einfache Lösung bietet, um ihre Vorräte zu verwalten und Verschwendung durch vergessene Mindesthaltbarkeitsdaten zu reduzieren. Die Anwendung ermöglicht die Erfassung von Artikeln mit Name, Mindesthaltbarkeitsdatum, Menge und Lagerort, bietet eine Haushalts-Verwaltung mit Einladungssystem, ein Dashboard mit Statistiken und Warnungen sowie ein flexibles Lizenz-System mit Free-, Basic- und Pro-Plänen.

Das Alleinstellungsmerkmal von FreshGuard liegt in der echten Multi-User-Haushaltsverwaltung mit einem durchdachten Rollensystem aus Eigentümer, Co-Eigentümer und Mitwirkenden. Automatische Warnungen mit farblicher Kennzeichnung in Grün, Gelb und Rot machen auf ablaufende Produkte aufmerksam. Als browserbasierte Anwendung ist keine Installation erforderlich und die Plattform läuft responsiv auf allen Geräten. Dabei ist FreshGuard vollständig DSGVO-konform mit Serverstandort in Deutschland.

Der Kundennutzen ist vielfältig: Weniger Lebensmittelverschwendung durch rechtzeitige Erinnerungen führt zu Kostenersparnis durch optimierten Verbrauch. Nutzer erhalten einen besseren Überblick über vorhandene Vorräte und können diese gemeinsam mit allen Haushaltsmitgliedern verwalten, was auch die Einkaufsplanung erleichtert und Zeit spart.

\section{Unternehmensziele}

Im ersten Jahr steht die Fertigstellung und Inbetriebnahme des MVP im Vordergrund, gefolgt von der Gewinnung erster Nutzer, dem Sammeln und Einarbeiten von Feedback sowie dem Erreichen von über 100 aktiven Nutzern. Langfristig, in den Jahren zwei bis fünf, soll die Nutzerbasis auf über 1.000 aktive Haushalte ausgebaut werden. Die Premium-Funktionen werden um Barcode-Scan zur schnellen Artikelerfassung, E-Mail-Erinnerungen an ablaufende Produkte und Rezeptvorschläge basierend auf vorhandenen Zutaten erweitert. Außerdem ist die Entwicklung einer mobilen App für iOS und Android geplant.

\section{Entwicklungsstand und Voraussetzungen}

Das MVP (Minimal Viable Product) mit allen Kernfunktionen ist fertiggestellt. Die Basisstruktur, die Haushalts-Verwaltung, das Lizenz-System und das Admin-Panel sind vollständig implementiert und abgeschlossen. Vor dem Start müssen noch das Produktiv-Hosting eingerichtet, finale Tests durchgeführt, die Benutzerdokumentation vervollständigt und Marketingmaterialien erstellt werden.

\section{Preisstrategie}

\begin{tabular}{llr}
    \toprule
    \textbf{Plan} & \textbf{Features} & \textbf{Preis} \\
    \midrule
    Free & 1 Haushalt, 50 Artikel, 1 Nutzer & 0 \euro{}/Monat \\
    Basic & 3 Haushalte, unbegrenzte Artikel, 5 Nutzer & 4,99 \euro{}/Monat \\
    Pro & Unbegrenzt, alle Features, Priority-Support & 9,99 \euro{}/Monat \\
    Pro Lifetime & Alle Pro-Features, einmalige Zahlung & 199,99 \euro{} \\
    \bottomrule
\end{tabular}

\vspace{0.5cm}
Die Vermarktung startet am \textbf{01. Februar 2026} nach Abschluss der Testphase.

\section{Rechtliche Aspekte}

Zu den erforderlichen gesetzlichen Formalitäten gehören die Gewerbeanmeldung, die Erstellung eines Impressums, eine DSGVO-konforme Datenschutzerklärung sowie die Formulierung allgemeiner Geschäftsbedingungen. Im Bereich der Schutzrechte ist geplant, den Markennamen \glqq FreshGuard\grqq{} beim DPMA schützen zu lassen. Die Domain freshguard.de ist bereits gesichert und der Quellcode wird als Geschäftsgeheimnis geschützt.
