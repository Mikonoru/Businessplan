\chapter{Chancen und Risiken}

\section{Chancen}

\subsection{Die drei größten Chancen}
\begin{enumerate}
    \item \textbf{Wachsendes Umweltbewusstsein:} Der gesellschaftliche Trend zu weniger Lebensmittelverschwendung und nachhaltigem Konsum stärkt die Nachfrage nach Lösungen wie FreshGuard.
    
    \item \textbf{Zunahme von Mehrpersonenhaushalten:} Wohngemeinschaften, Mehrgenerationenhäuser und größere Familien benötigen kollaborative Lösungen zur Haushaltsverwaltung.
    
    \item \textbf{Digitalisierung im Alltag:} Immer mehr Menschen nutzen digitale Tools zur Organisation ihres Alltags -- die Akzeptanz für Web-Apps steigt kontinuierlich.
\end{enumerate}

\subsection{Marktchancen}
\begin{itemize}
    \item Wachsender Markt für Nachhaltigkeits-Apps
    \item Steigende Sensibilität für Lebensmittelverschwendung (EU-Initiativen)
    \item Noch kein dominanter Marktführer im DACH-Raum
\end{itemize}

\subsection{Technologische Chancen}
\begin{itemize}
    \item Progressive Web Apps (PWA) ermöglichen App-ähnliche Erlebnisse
    \item Cloud-Technologien erlauben kosteneffiziente Skalierung
    \item Open-Source-Ökosystem reduziert Entwicklungskosten
\end{itemize}

\subsection{Strategische Chancen}
\begin{itemize}
    \item Kooperationen mit Nachhaltigkeits-Initiativen
    \item B2B-Erweiterung (Kantinen, kleine Gastro-Betriebe)
    \item Internationale Expansion nach erfolgreicher DACH-Etablierung
\end{itemize}

\section{Risiken}

\subsection{Die drei größten Risiken}
\begin{enumerate}
    \item \textbf{Konkurrenz:} Etablierte Apps (Bring!, NoWaste) haben größere Marketing-Budgets und bestehende Nutzerbasis.
    
    \item \textbf{Geringe Zahlungsbereitschaft:} Nutzer könnten die kostenlose Version bevorzugen und nicht auf Premium upgraden.
    
    \item \textbf{Technische Probleme:} Sicherheitsvorfälle oder Serverausfälle könnten das Vertrauen der Nutzer beschädigen.
\end{enumerate}

\subsection{Interne Risiken}
\begin{itemize}
    \item Ausfall eines Gründers (Krankheit, Ausstieg)
    \item Mangelnde Marketing-Erfahrung im Team
    \item Überlastung durch Mehrfachbelastung (Entwicklung + Betrieb)
\end{itemize}

\subsection{Externe Risiken}
\begin{itemize}
    \item Verschärfte Datenschutzvorschriften
    \item Wirtschaftliche Rezession reduziert Ausgabebereitschaft
    \item Technologischer Wandel (neue Plattformen/Standards)
\end{itemize}

\subsection{Absicherung gegen Risiken}
\begin{tabular}{p{5cm}p{8cm}}
    \toprule
    \textbf{Risiko} & \textbf{Gegenmaßnahme} \\
    \midrule
    Konkurrenz & Differenzierung durch USP (Multi-User, deutsches Hosting) \\
    Geringe Zahlungsbereitschaft & Flexibles Preismodell, günstige Einstiegspreise, Lifetime-Option \\
    Technische Probleme & Regelmäßige Backups, OWASP-Sicherheitsrichtlinien, SSL \\
    Gründer-Ausfall & Dokumentation, Vertretungsregelung, Wissenstransfer \\
    \bottomrule
\end{tabular}

\section{SWOT-Analyse}

\subsection{Stärken (Strengths)}
\begin{itemize}
    \item Technische Kompetenz des Gründerteams
    \item Echte Multi-User-Funktionalität (USP)
    \item DSGVO-Konformität mit deutschem Hosting
    \item Schlanke Kostenstruktur
    \item Fertiger MVP
\end{itemize}

\subsection{Schwächen (Weaknesses)}
\begin{itemize}
    \item Keine etablierte Marke/Nutzerbasis
    \item Begrenzte Marketing-Erfahrung
    \item Keine native App (nur Web)
    \item Begrenztes Startkapital
\end{itemize}

\subsection{Chancen (Opportunities)}
\begin{itemize}
    \item Wachsender Nachhaltigkeits-Trend
    \item Zunahme von Mehrpersonenhaushalten
    \item Digitalisierung des Alltags
    \item B2B-Potenzial (Kantinen, Gastro)
\end{itemize}

\subsection{Risiken (Threats)}
\begin{itemize}
    \item Etablierte Konkurrenz mit größerem Budget
    \item Geringe Zahlungsbereitschaft für Premium
    \item Technische Sicherheitsrisiken
    \item Wirtschaftliche Unsicherheiten
\end{itemize}
