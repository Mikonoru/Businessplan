\chapter{Rechtsform}

\section{Gewählte Rechtsform}
\textbf{GbR (Gesellschaft bürgerlichen Rechts)} zu Beginn.

Bei Wachstum ist eine Umwandlung in UG (haftungsbeschränkt) oder GmbH geplant.

\section{Begründung der Rechtsformwahl}

\subsection{Vorteile der GbR}
\begin{itemize}
    \item \textbf{Einfache Gründung:} Kein Notar, kein Handelsregistereintrag erforderlich
    \item \textbf{Geringe Kosten:} Nur Gewerbeanmeldung (ca. 30--50 \euro{})
    \item \textbf{Keine Mindesteinlage:} Kein Stammkapital erforderlich
    \item \textbf{Flexible Gestaltung:} Gesellschaftsvertrag frei verhandelbar
    \item \textbf{Schneller Start:} Gründung innerhalb weniger Tage möglich
\end{itemize}

\subsection{Nachteile der GbR}
\begin{itemize}
    \item Persönliche Haftung der Gesellschafter
    \item Nicht eintragungsfähig im Handelsregister
    \item Begrenzte Außenwirkung bei Geschäftspartnern
\end{itemize}

\subsection{Geplante Umwandlung}
Bei Erreichen von nachhaltigem Wachstum (ab ca. 50.000 \euro{} Jahresumsatz):
\begin{itemize}
    \item Umwandlung in \textbf{UG (haftungsbeschränkt)} oder
    \item Direkte Gründung einer \textbf{GmbH}
\end{itemize}

\section{Gesellschafteranteile und Haftung}

\subsection{Gesellschafterstruktur}
\begin{tabular}{llr}
    \toprule
    \textbf{Gesellschafter} & \textbf{Rolle} & \textbf{Anteil} \\
    \midrule
    Phillipp Atzler & Geschäftsführer & 40\% \\
    Maike Abel & Gesellschafterin & 30\% \\
    Elisa Dörnbrak & Gesellschafterin & 30\% \\
    \midrule
    \textbf{Gesamt} & & \textbf{100\%} \\
    \bottomrule
\end{tabular}

\subsection{Begründung der Anteilsverteilung}
\begin{itemize}
    \item Phillipp Atzler erhält 40\% als Geschäftsführer mit erhöhter Verantwortung
    \item Maike Abel und Elisa Dörnbrak erhalten je 30\% als gleichberechtigte Gesellschafterinnen
    \item Gleichgewicht verhindert Pattsituationen bei Abstimmungen
\end{itemize}

\subsection{Haftung}
Bei der GbR haften alle Gesellschafter persönlich und unbeschränkt mit ihrem Privatvermögen. Dies wird akzeptiert, da:
\begin{itemize}
    \item Das Geschäftsmodell risikoarm ist (Software-as-a-Service)
    \item Keine hohen Verbindlichkeiten geplant sind
    \item Bei Wachstum Umwandlung in haftungsbeschränkte Rechtsform erfolgt
\end{itemize}

\section{Gesellschaftsvertrag}

\subsection{Berücksichtigung der Gesellschafterinteressen}
\begin{itemize}
    \item GbR ermöglicht flexible Zusammenarbeit
    \item Alle Gesellschafter sind in Entscheidungen eingebunden
    \item Elisa Dörnbrak fungiert als Mediatorin bei Meinungsverschiedenheiten
\end{itemize}

\subsection{Wesentliche Regelungen im Gesellschaftsvertrag}
\begin{itemize}
    \item \textbf{Geschäftsführung:} Phillipp Atzler als alleinvertretungsberechtigter Geschäftsführer
    \item \textbf{Gewinnverteilung:} Entsprechend der Gesellschaftsanteile (40/30/30)
    \item \textbf{Entscheidungen:} Wichtige Entscheidungen mit 2/3-Mehrheit
    \item \textbf{Kündigung:} 6 Monate Kündigungsfrist für Gesellschafter
    \item \textbf{Wettbewerbsverbot:} Während der Gesellschaftszugehörigkeit
    \item \textbf{Streitbeilegung:} Mediation vor Gerichtsverfahren
\end{itemize}
